\chapter{Filtro Pasa Bajos}
Se diseñaron filtros pasa bajos para funcionar como el Filtro Anti Alias (FAA) y el Filtro Recuperador (FR) en el esquema a implementar.
\section{Especificaciones del Filtro}
Se pidió que el Filtro Pasa Bajos cumpla con las siguientes especificaciones:

\begin{table}[ht]
    \centering
    \begin{tabular}{|c|c|c|}
        \hline
        $f_a$  &   $A_p$   &   $A_a$    \\
        \hline
        $1.5 f_p$   &   $1 \si{\deci\bel}$ &   $41 \si{\deci\bel}$    \\
        \hline
    \end{tabular}
    \caption{Especificaciones de los Filtros Pasa Bajo}
\end{table}

Se eligió un $fp = 50 \si{\kilo\hertz}$ dado que las señales que serán muestreadas son \dots

Usando un programa de diseño de filtros, se obtuvo que para estas especificaciones, utilizando la aproximación de Legendre, se obtiene un filtro de orden $n=8$. Para su implementación, se utilizaron $4$ filtros de segundo orden en cascada.

Con la información obtenida del programa de diseño de filtros, se conoció que las características de cada celda implementada debería cumplir con las siguientes características:

\begin{table}[ht]
    \centering
    \begin{tabular}{|c|c|c|}
        \hline
        N° Celda    & $f_0(\si{\kilo\hertz})$ & $Q_0$ \\
        \hline
        1   & 51.05 & 6.04  \\
        2   & 44.37 & 1.84  \\
        3   & 33.69 & 0.91  \\
        4   & 24.40 & 0.54  \\
        \hline
    \end{tabular}
\end{table}

\section{Realización del Filtro}

Para implementar el filtro pasa bajos se utilizaron celdas Rauch de construcción pasa bajos. Su esquema se muestra en la Figura \ref{fig:Rauch-Cell}.

\begin{figure}[ht]
    \centering
    \begin{circuitikz}[american voltages]
\draw
(0,0) node[op amp] (opamp) {}
(opamp.+) node[ground]{}
(opamp.-) to[resistor=$R_2$,*-*] ++(-3,0) coordinate(tmp)
(tmp) to[capacitor=$C_1$] ++(0,-2) node[ground]{}
(tmp) to[resistor=$R_1$,-o] ++(-3,0) node[left]{$V_{in}$}
(opamp.out) to[short] ++(0,2) coordinate(tmp1) to [short] ++(0,1) coordinate(tmp2)
(tmp1) to[capacitor=$C_2$] ++(-2,0) -| (opamp.-)
(tmp2) to[resistor,l_=$R_3$] ++(-2,0) -| (tmp)
(opamp.out) to[short,-o] ++(1,0) node[right]{$V_{out}$}
;
\end{circuitikz}
    \caption{Celda Rauch}
    \label{fig:Rauch-Cell}
\end{figure}

La transferencia total de esta celda estará dada por la expresión:
\begin{equation}
    H(s)=\frac{R_3}{R_1}\cdot\frac{1/R_3 R_2 C_1 C_2}{s^2+s \frac{1}{C_1}(1/R_1+1/R_2+1/R_3)+1/R_2 R_3 C_1 C_2}
\end{equation}

Comparando con la expresión general de la transferencia de un filtro pasabajos de segundo orden:

\begin{equation*}
    H(s)=\frac{K \omega_0^2}{s^2 +s\frac{\omega_0}{Q_0}+\omega_0 ^2}
\end{equation*}

Se obtiene que:
\begin{align}
    K   &=  \frac{R_3}{R_1}\\
    \omega_0^2  &=  \frac{1}{R_2 R_3 C_1 C_2}\\
    \frac{\omega_0}{Q_0}    &=  \frac{1}{C_1}\left(\frac{1}{R_1}+\frac{1}{R_2}+\frac{1}{R_3}\right)
\end{align}

Dado que se desea que la ganancia en continua sea $H(0)=0\si{\deci\bel}$, se puede tomar $R_3 = R_1 = R$. Se definió entonces también $R_2=a^2R$ y $C_1 = k^2 C_2 = k^2 C$. Por lo tanto, la transferencia de la celda resulta:

\begin{align}
    K&=1\\
    \omega_0^2&=\frac{1}{a^2 R^2 k^2 C^2} \Rightarrow \omega_0 =\frac{1}{ka}\cdot\frac{1}{R C}\\
    \frac{\omega_0}{Q_0} &= \frac{1}{k^2 C} \left(\frac{2}{R}+\frac{1}{a^2 R}\right) \label{eq:filterwo/Q0}
\end{align}

Operando sobre \eqref{eq:filterwo/Q0} se obtiene

\begin{align*}
    \frac{1}{Q_0} &= \frac{1}{k^2 a^2 RC}\left(2 a^2 + 1 \right)\cdot kaRC\\
    \frac{1}{Q_0} &= \frac{1}{ka} \left(2a^2+1\right)\\
    Q_0 &= \frac{ka}{2a^2+1}
\end{align*}

Por lo tanto la expresión del factor de calidad de la celda $Q_0$ estará dada por
\begin{equation}
    Q_0 = \frac{ka}{2a^2+1}
\end{equation}

Entonces, definiendo nuestras resistencias y capacidades base (TOMAR UN CRITERIO) se pueden calcular las relaciones entre resistencias y capacidades entre cada componente:

\begin{align*}
    ka &= \frac{1}{\omega_0 RC}\\
    a^2&= 0.5\left(\frac{ka}{Q_0}-1\right)\\
    k^2&= \frac{1}{\omega_0^2 a^2 R^2 C^2}
\end{align*}

En primer lugar, se definió como resistencia base $R=1\si{\kilo\ohm}$ y como capacitancia base $C=100\si{\pico\farad}$. Luego, teniendo en cuenta los valores y expresiones anteriores, se obtuvo el valor necesario para cada componente.

\begin{table}[ht]
    \begin{center}
        \begin{tabular}{|c|c|c|c|c|c|c|c|}
            \hline
            Celda   &  $f_0(\si{\kilo\hertz})$   &   $Q_0$   &   $R_1 (\si{\kilo\ohm})$&$R_2(\si{\kilo\ohm})$&$R_3(\si{\kilo\ohm})$&$C_1(\si{\nano\farad})$&$C_2(\si{\pico\farad})$   \\
            \hline
            1   & 51.05 & 6.04  & 1.00  & 2.08  & 1.00  & 46.7  & 100   \\
            2   & 44.37 & 1.84  & 1.00  & 9.25  & 1.00  & 13.9  & 100   \\
            3   & 33.69 & 0.91  & 1.00  & 25.5  & 1.00  & 8.77  & 100   \\
            4   & 24.40 & 0.54  & 1.00  & 59.9  & 1.00  & 7.10  & 100   \\
            \hline
        \end{tabular}
    \end{center}
    \caption{Valores de los componentes para cada una de sus celdas}
\end{table}
